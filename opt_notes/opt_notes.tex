\documentclass[12pt,a4paper]{article}

\usepackage[top=3cm,bottom=3cm,left=3cm,right=3cm]{geometry}
%% Packages
%% ========

%% LaTeX Font encoding -- DO NOT CHANGE
\usepackage[OT1]{fontenc}

%% Babel provides support for languages.  'english' uses British
%% English hyphenation and text snippets like "Figure" and
%% "Theorem". Use the option 'ngerman' if your document is in German.
%% Use 'american' for American English.  Note that if you change this,
%% the next LaTeX run may show spurious errors.  Simply run it again.
%% If they persist, remove the .aux file and try again.
\usepackage[english]{babel}

%% Input encoding 'utf8'. In some cases you might need 'utf8x' for
%% extra symbols. Not all editors, especially on Windows, are UTF-8
%% capable, so you may want to use 'latin1' instead.
\usepackage[utf8]{inputenc}

%% This changes default fonts for both text and math mode to use Herman Zapfs
%% excellent Palatino font.  Do not change this.
%\usepackage[sc]{mathpazo}

%% The AMS-LaTeX extensions for mathematical typesetting.  Do not
%% remove.
\usepackage{amsmath,amssymb,amsfonts,amsthm,mathrsfs}

\usepackage[dvipsnames]{xcolor}
\usepackage{tcolorbox}

%% LaTeX' own graphics handling
\usepackage{graphicx}

%% This allows you to add .pdf files. It is used to add the
%% declaration of originality.
\usepackage{pdfpages}

\usepackage{mathtools}
\usepackage{booktabs}

\numberwithin{equation}{section}

\newtheoremstyle{mystyle}% ⟨name ⟩ 
{3pt}% ⟨Space above ⟩1 
{3pt}% ⟨Space below ⟩1
{}% ⟨Body font ⟩
{}% ⟨Indent amount ⟩2
{\sffamily}% ⟨Theorem head font⟩
{.}% ⟨Punctuation after theorem head ⟩
{.5em}% ⟨Space after theorem head ⟩3
{}% ⟨Theorem head spec (can be left empty, meaning ‘normal’)⟩
%
%
%\newtheoremstyle{break}%
%{}{}%
%{}{}%
%{\bfseries}{}%  % Note that final punctuation is omitted.
%{\newline}{}

\theoremstyle{mystyle}

\newtheorem{definition}{Definition}[section]
\newtheorem{theorem}[definition]{Theorem}
\newtheorem{lemma}[definition]{Lemma}
\newtheorem{corollary}[definition]{Corollary}
\newtheorem{proposition}[definition]{Proposition}
\newtheorem{example}[definition]{Example}
%\tcbuselibrary{theorems}
\tcbuselibrary{skins,breakable}



\tcolorboxenvironment{theorem}{
	enhanced jigsaw,colframe=Salmon!90!Black,interior hidden, breakable,before skip=10pt,after skip=10pt 
}

\tcolorboxenvironment{proposition}{
	blanker,breakable,left=5mm,
	before skip=10pt,after skip=10pt,
	borderline west={1mm}{0pt}{Green!70}
}

\tcolorboxenvironment{definition}{
	blanker,breakable,left=5mm,
	before skip=10pt,after skip=10pt,
	borderline west={1mm}{0pt}{cyan!40!black}
}

\tcolorboxenvironment{example}{
	blanker,breakable,left=5mm,
	before skip=10pt,after skip=10pt,
	borderline west={1mm}{0pt}{green!35!black}
}

\tcolorboxenvironment{lemma}{
	blanker,breakable,left=5mm,
	before skip=10pt,after skip=10pt,
	borderline west={1mm}{0pt}{RoyalPurple!55!Aquamarine!100!}
}

\tcolorboxenvironment{corollary}{
	blanker,breakable,left=5mm,
	before skip=10pt,after skip=10pt,
	borderline west={1mm}{0pt}{CornflowerBlue!60!Black}
}




\tcolorboxenvironment{proof}{% `proof' from `amsthm' 
	blanker,breakable,right=5mm,
	before skip=10pt,after skip=10pt,
	borderline east={0.5mm}{1pt}{red!10!white}}

\usepackage[linkcolor=blue,colorlinks=cyan,citecolor=red,filecolor=black]{hyperref}

\usepackage{hyperref}
\hypersetup{
	colorlinks = true,
	linkcolor=NavyBlue,
	citecolor=OrangeRed,
	filecolor=orange}

\usepackage{tikz}
\usetikzlibrary{calc}
\usetikzlibrary{cd}



\title{Notes on Optimization}
\author{Liu Zhizhou}
\date{First Created: August 6, 2022\\
	Last Modified: \today}


\newcommand{\R}{\mathbb{R}}
\newcommand{\N}{\mathbb{N}}
\newcommand{\grad}{\nabla}
\renewcommand{\d}{\mathrm{d}}
\newcommand{\T}{\top}


\newcommand{\diag}{\operatorname{diag}}
\newcommand{\conv}{\operatorname{conv}}
\newcommand{\cone}{\operatorname{cone}}
\newcommand{\aff}{\operatorname{aff}}

\newcommand{\abs}[1]{\left\lvert #1 \right\rvert}
\newcommand{\norm}[1]{\left\lVert #1 \right\rVert}
\newcommand{\inner}[1]{\left\langle #1 \right\rangle}
\begin{document}
	{\sffamily \maketitle}
	
	\cite{Bertsekas/99} is brilliant textbook for optimization, which is also the main reference of this notes.
	
	\tableofcontents

	\section{Derivatives}
	Let $f:\R^n \to \R$. The \emph{gradient} of $f$ at $x$ is defined as the column vector
	$$
	\grad f(x)=
	\begin{bmatrix}
		\frac{\partial f(x)}{\partial x_1}\\
		\vdots\\
		\frac{\partial f(x)}{\partial x_n}
	\end{bmatrix}.
	$$
	If $f$ is a vector-valued function, i.e. $f:\R^n\to \R^m$, with component functions $f_1,\dots,f_m$, then
	$$
	\grad f(x)=
	\begin{bmatrix}
		\grad f_1(x) & \cdots & \grad f_m(x)
	\end{bmatrix}.
	$$
	The transpose of $\grad f$ is called the \emph{Jacobian} of $f$. The Jacobian of $f$ is the matrix whose $ij$-th entry is equal to the partial derivative $\frac{\partial f_i}{\partial x_j}$.
	
	The \emph{Hessian} of $f:\R^n\to \R$ is the matrix whose $ij$-th entry is equal to $\frac{\partial^2 f}{\partial x_i \partial x_j}$, denoted by $\grad^2 f$.
	
	Be careful that, for $f:\R^n \to \R$, $\grad^2 f\neq \grad(\grad f)$, but $\grad^2 f = \grad(\grad f^\T)$.
	\begin{proposition}[chain rule]
		Let $f:\R^k\to \R^m$ and $g:\R^m\to\R^n$ be smooth functions, and $h=g(f(x))$. Then
		$$
		\grad h(x) = \grad f(x)\grad(g(f(x)))
		$$
		for all $x\in \R^k$.
	\end{proposition}
	Some useful relations:
	\begin{enumerate}
		\item $\grad (Ax)=A^\T$;
		\item $\grad (x^\T A x)=(A+A^\T)x$; in particular, if $Q$ is symmetric, then $\grad(x^\T Q x)=2Qx$ and $\grad(\norm{x}^2)=\grad(x^\T x)=2x$;
		\item $\grad(f(Ax))=A^\T \grad f(Ax)$;
		\item $\grad^2(f(Ax))=A^\T \grad^2 f(Ax)A$;
	\end{enumerate}
	The shape of the left hand side would be helpful to memorize the right hand side.
	
	\begin{theorem}[Second Order Taylor Expansions]
		Let $f:\R^n \to \R$ be twice continuously differentiable over an open sphere $S$ centered at a vector $x$. Then for all $d$ such that $x+d\in S$,
		\begin{enumerate}
			\item we have
			$$
			f(x+d)=f(x)+d^\T \grad f(x) +\frac{1}{2}d^\T\left(\int_0^1\left(\int_0^\T \grad^2 f(x+\tau d)\d \tau\right)\d t\right)d.
			$$
			\item there exists
			$$
			f(x+d)=f(x)+d^\T \grad f(x)+\frac{1}{2}d^\T \grad^2 f(x+\alpha d)d.
			$$
			\item there holds
			$$
			f(x+d)=f(x)+d^\T \grad f(x)+\frac{1}{2}d^\T \grad^2 f(x)d+o(\norm{d}^2).
			$$
		\end{enumerate}
	\end{theorem}


	\section{Convex Sets and Functions}
	\subsection{Convexity}
	\begin{definition}[convex set, convex function]
		A subset $C$ of $\R^n$ is called \emph{convex} if 
		$$
		\alpha x+(1-\alpha)y\in C
		$$ for all $x,y\in C$ and $\alpha\in [0,1]$. A function $f:C\to \R$ is called \emph{convex} if 
		\begin{equation}
			f(\alpha x+(1-\alpha )y)\leq 
			\alpha f(x)+(1-\alpha)f(y)\label{eq:convex}
		\end{equation}
		for $x,y\in C$ and $\alpha\in [0,1]$. The function is called \emph{concave} if $-f$ is convex.
	\end{definition}
	\begin{definition}[strictly convex]
		The function $f$ is called \emph{strictly convex} if Eq.(\ref{eq:convex}) is strict for all $x\neq y$ and $\alpha\in(0,1)$.
	\end{definition}

	\begin{proposition}[First Derivative Characterizations]
		Let $C$ be a convex subset of $\R^n$ and let $f:\R^n \to \R$ be differentiable over $\R^n$. Then
		\begin{enumerate}
			\item $f$ is convex over $C$ if and only if
			\begin{equation}
				f(z)\geq f(x)+(z-x)^\T \grad f(x)
			\end{equation}
			for all $x,z\in C$.
			\item $f$ is strictly convex over $C$ if and only if the above inequality is strict whenever $x\neq z$.
		\end{enumerate}
	\end{proposition}

	
	\begin{definition}[strongly convex]
		A function $f:\R^n \to \R$ is called \emph{strongly convex} if for some $\sigma>0$, we have
		\begin{equation}
			f(y)\geq f(x)+\grad f(x)^\T (y-x)+\frac{\sigma}{2}\norm{x-y}^2
		\end{equation}
		for all $x,y\in \R^n$.
	\end{definition}
	It can be shown that an equivalent definition is that
	\begin{equation}
		(\grad f(x)-\grad f(y))^\T(x-y)\geq \sigma \norm{x-y}^2
	\end{equation}
	for all $x,y\in \R^n$.
	
	
	\subsection{Convex Hull and Affine Hull}
	\begin{definition}[convex combination, convex hull]
		A \emph{convex combination} of elements of $X$ is a vector of the form $\sum_{i=1}^m \alpha_i x_i$, where $x_i\in X$ and $\alpha_i\in \R$ such that $\alpha_i\geq 0$ for $i=1,\dots,m$ and $\sum_{i=1}^m \alpha_i =1$.
		
		The \emph{convex hull} of $X$, denoted $\conv{X}$, is the set of all convex combinations of elements of $X$, i.e.
		\begin{equation}
			\conv(X) = \left\{\sum_{i\in I} \alpha_i x_i: \alpha_i\geq 0, \sum_{i\in I}\alpha_i=1, x_i\in X, I\subseteq \N \right\}.
		\end{equation} 
	\end{definition}
	
	Recall that a \emph{linear manifold} (or linear affine) is a set of the form $x+S$, where $S$ is a subspace.
	\begin{definition}[affine hull]
		If $S\subset \R^n$, the \emph{affine hull} of $S$, denoted $\aff(S)$, is the intersection of all linear manifolds containing $S$. 
	\end{definition}
	
	\begin{definition}[cone]
		A set $C\subset \R^n$ is said to be \emph{cone} if $ax\in C$ for all $a\geq 0$ and $x\in C$. The \emph{cone generated by $X$}, denoted $\cone(X)$, is the set of all nonnegative combinations of elements of $X$.
	\end{definition}
	
	
	

	\section{Main Optimality Conditions}
	\begin{theorem}[Necessary Optimality Conditions]
		Let $x^*$ be an unconstrained local minimum of $f:\R^n \to \R$, and assume that $f$ is continuously differentiable in an open set $S$ containing $x^*$. Then we have the \emph{First Order Necessary Condition}:
		\begin{equation}
			\grad f(x^*)=0.
		\end{equation}
		If in addition $f$ is twice continuously differentiable within $S$, then we have the \emph{Second Order Necessary Condition}:
		\begin{equation}
			\grad^2 f(x^*)\succeq 0.
		\end{equation}
	\end{theorem}
	The intuition of this theorem is considering
	$$
	f(x^*+\Delta x)-f(x^*)\approx \grad f(x^*)^\T \Delta x,
	$$
	and similarly for second order,
	$$
	f(x^*+\Delta x)-f(x^*)\approx \grad f(x^*)^\T \Delta x +\frac{1}{2}\Delta x^\T \grad^2 f(x^*)\Delta x.
	$$
	Read rigorous proof to see the reason.
	\begin{proof}
		Fix some $d\in \R^n$. Consider $g(\alpha)\triangleq f(x^*+\alpha d)$. Then
		$$
		0\leq 
		\lim_{\alpha\to 0}\frac{f(x^*+\alpha d)-f(x^*)}{\alpha} 
		= \frac{\d g}{\d \alpha}(0)=d^\T \grad f(x^*).
		$$
		The "$\leq$" is because $x^*$ is the local minimum. Replace $d$ by $-d$, then it must be $\grad f(x^*)=0$.
		
		Assume $f$ is twice differentiable. Then the second order expansion of $g(\alpha)$ in $\alpha=0$ yields
		$$
		g(\alpha) = g(0) + \frac{\d g}{\d \alpha}(0)\alpha +
		\frac{1}{2} \frac{\d^2 g}{\d \alpha^2}(0) \alpha^2 + o(\alpha^2).
		$$
		Equivalently,
		$$
		f(x^*+\alpha d) - f(x^*) = d^\T \grad f(x^*)\alpha + \frac{\alpha^2}{2} d^\T \grad^2 f(x^*) d + o(\alpha^2).
		$$
		Since $\grad f(x^*)=0$, for $\alpha$ positive and near 0, we have
		$$
		0\leq \frac{f(x^*+\alpha d)-f(x^*)}{\alpha^2}= 
		\frac{1}{2} d^\T \grad^2 f(x^*) d + \frac{o(\alpha^2)}{\alpha^2}.
		$$
		Then let $\alpha\to 0$, we obtain $d^\T \grad^2 f(x^*) d\geq 0$, which means $\grad^2 f(x^*) \succeq 0$.
	\end{proof}
	
	\begin{proposition}
		If $X$ is a convex subset of $\R^n$ and $f:\R^n \to \R$ is convex over $X$, then a local minimum of $f$ is also a global minimum. If in addition $f$ is strictly convex over $X$, then $f$ has at most one global minimum over $X$. Moreover, if $f$ is strongly convex and $X$ is closed, then $f$ has a unique global minimum over $X$.
	\end{proposition}
	
	\begin{theorem}[Convex Case - Necessary and Sufficient Conditions]
		Let $X$ be a convex set and let $f:\R^n \to \R$ be a convex function over $X$. Then
		\begin{enumerate}
			\item If $f$ is continuously differentiable, then
			$$
			\grad f(x^*)^\T(x-x^*)\geq 0
			$$
			for all $x\in X$ is a necessary and sufficient condition for $x^*$ to be a global minimum of $f$ over $X$.
			\item If $X$ is open and $f$ is continuously differentiable over $X$, then $\grad f(x^*)=0$ is a necessary and sufficient condition for $x^*$ to be a global minimum of $f$ over $X$.
		\end{enumerate}
	\end{theorem}
	Note that in the second statement, we require $X$ to be open.

	The intuition of this theorem is also
	$$
	f(x^*+\Delta x)-f(x^*)\approx \grad f(x^*)^\T \Delta x.
	$$
	The proof of this need the first order characterization of convexity,
	$$
	f(x)\geq f(x^*) +\grad f(x^*)^\T (x-x^*)
	$$
	for all $x\in X$.
	
	A geometric illustration of $\grad f(x^*)^\T(x-x^*)$ is that: $\grad f(x^*)$ is the direction that $f$ increase the most, the condition means that the connection of $x^*$ and all feasible points $x$ in $X$ has angle less than $\frac{\pi}{2}$ with the gradient; in other words, all the direction makes $f$ increase.
	
	\begin{theorem}[Second Order Sufficient Optimality Conditions]
		Let $f:\R^n \to \R$ be twice continuously differentiable over an open set $S$. Suppose that a vector $x^*\in S$ satisfies the conditions: (i) $\grad f(x^*)=0$ and (ii) $\grad^2 f(x^*)\succ 0$. Then $x^*$ is a strict unconstrained local minimum of $f$. In particular, there exists scalars $\gamma>0$ and $\epsilon>0$ such that
		$$
		f(x)\geq f(x^*)+\frac{\gamma}{2}\norm{x-x^*}^2
		$$
		for all $\norm{x-x^*}<\epsilon$.
	\end{theorem}
	\begin{proof}
		Denote $\lambda$ the smallest eigenvalue of $\grad^2 f(x^*)$. Since $\grad^2 f(x^*)\succ 0$, $\lambda >0$. We have $d^\T \grad^2 f(x^*)d \geq \lambda \norm{d}^2$ for all $d\in \R^n$. By the second order Taylor expansion
		\begin{align*}
			f(x^*+d)-f(x^*)&=\grad f(x^*)^\T d + \frac{1}{2} d^\T \grad^2 f(x^*)d+o(\norm{d}^2)\\
			&\geq \frac{\lambda}{2}\norm{d}^2+o(\norm{d}^2)\\
			&=\left(\frac{\lambda}{2}+\frac{o(\norm{d}^2)}{\norm{d}^2}\right)\norm{d}^2.
		\end{align*}
		Then choose $\epsilon>0$ and $\gamma>0$ such that for $\norm{d}<\epsilon$,
		$$
		\frac{\lambda}{2}+\frac{o(\norm{d}^2)}{\norm{d}^2}\geq \frac{\gamma}{2}.
		$$
		Then the proof is complete.
	\end{proof}



	\section{Algorithms: Gradient Methods}
	Optimality conditions often provide the basis for the development and the analysis of the algorithms. The idea of the algorithms rely on an important idea, called \emph{iterative descent}, i.e. $f(x^{k+1})<f(x^k)$. Gradient methods, also called gradient descent methods, implement the idea of iterative descent. The iteration is 
	\begin{equation}
		x^{k+1} = x^k + \alpha^k d^k,
	\end{equation}
	where $k=0,1,\dots$, $\grad f(x^k)^\T d^k<0$, $\alpha^k\in \R$ and $d^k\in \R^n$. There is a large variety of possibilities for choosing direction $d^k$ and stepsize $\alpha^k$.
	
	\subsection{Descent Direction}
	To make sure $\grad f(x^k)^\T d^k<0$, most gradient methods take the form $d^k =-D^k \grad f(x^k)$, where $D^k$ is a positive definite symmetric matrix. The iteration becomes
	\begin{equation}
		x^{k+1} = x^k -\alpha^k D^k \grad f(x^k).
	\end{equation}
	Different choice of $D^k$ result in different methods, see Table \ref{table:descent direction}.
	\begin{table}[]
		\centering
		\begin{tabular}{@{}cc@{}}
			\toprule
			Name of Method                     & Choice of $D^k$                                                                                                                                 \\ \midrule
			Steepest Descent                   & $D^k = I$                                                                                                                                       \\
			Newton's Methods                   & $D^k=\left(\grad^2 f(x^k)\right)^{-1}$                                                                                                          \\
			Diagonally Scaled Steepest Descent & $D^k=\diag(d_1^k,\dots,d_n^k)$                                                                                                                  \\
			Modified Newton's Method           & $D^k=\left(\grad^2 f(x^0)\right)^{-1}$                                                                                                          \\
			Gauss Newton Method                & $D^k=(\grad g(x^k)\grad g(x^k)^\T)^{-1}$\\ \bottomrule
		\end{tabular}
	\caption{
		Various choice of the positive definite matrix $D^k$, where $d^k=-D^k \grad f(x^k)$. Gauss Newton Method is widely used when the cost function $f(x)$ is of the form 
		$f(x)=\frac{1}{2}\norm{g(x)}^2=\frac{1}{2}\sum_{i=1}^m (g_i(x))^2,$
		where $g=(g_1,\dots,g_m)$, which is a problem often encountered in statistical data analysis and in the context of neural network training.}
	\label{table:descent direction}
	\end{table}
	
	\subsection{Stepsize}
	There are a number of rules for choosing the stepsize $\alpha^k$ in a gradient method. We give some that are used widely in practice in Table \ref{table:stepsize}.
	% Please add the following required packages to your document preamble:
	% \usepackage{booktabs}
	\begin{table}[]
		\centering
		\begin{tabular}{@{}cc@{}}
			\toprule
			Name of Method            & Choice of $\alpha^k$                                       \\ \midrule
			Minimization Rule         & $\alpha^k = \arg\min_{\alpha\geq 0} f(x^k+\alpha d^k)$     \\
			Limited Minimization Rule & $\alpha^k = \arg\min_{\alpha\in [0,s]} f(x^k+\alpha d^k)$  \\
			Armijo Rule               & $\alpha^k=\beta^{m_k}s$                                    \\
			Constant Stepsize         & $\alpha^k =s$                                              \\
			Diminishing Stepsize      & $\alpha^k\to 0$, where $\sum_{k=0}^\infty \alpha^k=\infty$ \\ \bottomrule
		\end{tabular}
	\caption{Various choice of stepsize $\alpha^k$. In Arimijo rule, first choose fix scalars $s,\beta$ and $\sigma$, with $0<\beta<1$ and $0<\sigma<1$, let $m_k$ be the first non-negative integer $m$ such that $f(x^k)-f(x^k+\beta^m s d^k)\geq -\sigma \beta^m s \grad f(x^k)^\T d^k$. Note that here $\beta^m$ means $\beta$ to the $m$-th power. In diminishing method, we require $\sum_{k=0}^{\infty}\alpha^k=\infty$ to guarantees that $\{x^k\}$ does not converge to a non-stationary point. Indeed, if $x^k\to\bar{x}$, then for large $m,n$, $x^m\approx x^n \approx \bar{x}$, also $x^m\approx x^n -(\sum_{k=n}^{m-1}\alpha^k)\grad f(\bar{x})$, which shows $\grad f(\bar{x})$ must be zero.}
	\label{table:stepsize}
	\end{table}
	
	
	
	\subsection{Mathematical Statements for Convergence Results}
	There is a common line of proof for the convergence results. The main idea is that the cost function is improved at each iteration and the improvement is ``substantial'' near a non-stationary point, i.e. it is bounded away from zero. We then argue that the algorithm cannot approach a non-stationary point, since in this case the total cost improvement would accumulate to infinity.
	
	I use ``Proposition'' instead of ``Theorem'' for the convergence results because in my opinion they are just some theoretic backgrounds (results) we should know in applications of the algorithms.
	
	\begin{definition}[gradient related]
		We say that the direction $\{d^k\}$ is \emph{gradient related} to $\{x^k\}$ if for any subsequence $\{x^k\}_{k\in \mathcal{K}}$ that converges to a non-stationary point, the corresponding subsequence $\{d^k\}_{k\in \mathcal{K}}$ is bounded and satisfies
		\begin{equation}
			\limsup_{k\to\infty,k\in \mathcal{K}}\grad f(x^k)^\T d^k<0.
		\end{equation}
	\end{definition}
	
	\begin{proposition}[Stationary of Limit Points]
		Let $\{x_k\}$ be a sequence generated by a gradient method $x^{k+1}=x^k + \alpha^k d^k$ and assume that $\{d^k\}$ is gradient related and $\alpha^k$ is chosen by the minimization rule, or the limited minimization rule, or the Armijo rule. Then every limit point of $\{x^k\}$ is a stationary point.
	\end{proposition}
	\begin{proof}
		The key to prove this is noting that if $\{x^k\}$ converges to a non-stationary point, by the assumption of gradient related property of $\{d^k\}$, and by the definition of Armijo rule, we must have for some index $\bar{k}\geq 0$,
		$$
		f(x^k)-f(x^k+(\alpha^k/\beta)d^k)<-\sigma(\alpha^k/\beta)\grad f(x^k)^\T d^k,
		$$
		for all $k\in \mathcal{K}$ and $k\geq \bar{k}$; in other words, if $\alpha^k\to 0$ there would exist a slight larger $\alpha^k/\beta$ that does not make ``substantial'' movement, which will result in contradiction.
	
		For minimization rule, just note that
		$$
		f(x^k)-f(x^{k+1})\geq f(x^k)-f(\tilde{x}^{k+1})\geq -\sigma(\tilde{\alpha})\grad f(x^k)^\T d^k,
		$$
		where $\{\tilde{x}\}$ is the sequence generated via Armijo rule and $\{\tilde{\alpha}\}$ is the corresponding stepsize. Then repeat the method to reach contradiction.
	\end{proof}

	\begin{lemma}[Descent Lemma]
		Let $f:\R^n \to \R$ be continuously differentiable, and let $x$ and $y$ be two vectors in $\R^n$. Suppose that
		$$
		\norm{\grad f(x+ty)-\grad f(x)}\leq L t \norm{y}
		$$
		for all $t\in [0,1]$, where $L$ is some scalar. Then
		$$
		f(x+ty)\leq f(x)+y^\T \grad f(x) +\frac{L}{2}\norm{y}^2.
		$$
	\end{lemma}
	We can interpret this lemma as: if $f$ is Lipschitz continuous then the variation of $f(x)$ is controlled by its slope plus $\frac{L}{2}\norm{y}^2$. In fact, Lipschitz condition requires roughly that the ``curvature'' of $f$ is no more than $L$ at all points and in all directions.
	
	\begin{proposition}[Constant Stepsize]
		Let $\{x^k\}$ be a sequence generated by a gradient method $x^{k+1}=x^k +\alpha^k d^k$, where $\{d^k\}$ is gradient related. Assume that $f$ is $L$-Lipschitz continuous and that for all $k$ we have $d^k\neq 0$ and 
		\begin{equation}
			\epsilon \leq \alpha^k\leq (2-\epsilon)\bar{\alpha}^k, \label{eq:constant stepsize condition}
		\end{equation}
		where
		$$
		\bar{\alpha}^k =\frac{\abs{\grad f(x^k)^\T d^k}}{L\norm{d^k}^2},
		$$
		and $\epsilon\in(0,1]$ is a fixed scalar. Then every limit point of $\{x^k\}$ is a stationary point of $f$.
	\end{proposition}
	The intuition of the proof is that by descent lemma, $f(x^k)-f(x^k +\alpha^k d^k)$ is bounded by a quadratic overestimation.
	\begin{proof}
		By using descent lemma combined with the right hand side of Eq. (\ref{eq:constant stepsize condition}), we have
		$$
		f(x^k)-f(x^k +\alpha^k d^k) \geq \frac{1}{2}\epsilon^2 \abs{\grad f(x^k)^\T d^k}.
		$$
	\end{proof}
	In the case of steepest descent, the condition on stepsize becomes
	$$
	\epsilon\leq \alpha^k \leq \frac{2-\epsilon}{L}.
	$$
	\begin{proposition}[Diminishing Stepsize]
		Let $\{x^k\}$ be a sequence generated by a gradient method $x^{k+1}=x^k +\alpha^k d^k$. Assume that $f$ is $L$-Lipschitz continuous and there exist positive scalars $c_1,c_2$ such that for all $k$ we have
		\begin{equation}
			c_1 \norm{\grad f(x^k)}^2 \leq -\grad f(x^k)^\T d^k,\quad \norm{d^k}^2 \leq c_2 \norm{\grad f(x^k)}^2.
		\end{equation}
		Suppose also that
		$$
		\alpha^k\to 0, \quad \sum_{k=0}^\infty \alpha^k =\infty.
		$$
		Then either $f(x^k)\to -\infty$ or else $\{f(x^k)\}$ converges to a finite value and $\grad f(x^k)\to 0$. Furthermore, every limit point of $\{x^k\}$ is a stationary point of $f$.
	\end{proposition}
	\begin{proof}
		Tedious. First use descent lemma to show $\liminf_{k\to\infty}\norm{\grad f(x^k)}=0$. Then separate the sequence into $\norm{\grad f(x^k)}>\epsilon/3$ and $\norm{\grad f(x^k)}\leq \epsilon/3$ to reach contradiction.
	\end{proof}
	
	\begin{theorem}[Capture Theorem]
		Let $f$ be continuously differentiable and let $\{x^k\}$ be a sequence satisfying $f(x^{k+1})\leq f(x^k)$ for all $k$ and generated by a gradient method $x^{k+1}=x^k +\alpha^k d^k$, which is convergent in the sense that every limit point of sequences that it generates is a stationary point of $f$. Assume that there exist scalars $s>0$ and $c>0$ such that for all $k$ there holds
		$$
		\alpha^k \leq s,\quad \norm{d^k}\leq c\norm{\grad f(x^k)}.
		$$
		Let $x^*$ be a local minimum of $f$, which is the only stationary point of $f$ within some open set. Then there exists an open set $S$ containing $x^*$ such that if $x^{\bar{k}}\in S$ for some $\bar{k}\geq 0$, then $x^k\in S$ for all $k\geq \bar{k}$ and $\lim_{k\to\infty}x^k=x^*$. Furthermore, given any scalar $\bar{\epsilon}>0$, the set $S$ can be chosen so that $\norm{x-x^*}<\bar{\epsilon}$ for all $x\in S$.
	\end{theorem}
	
	\subsection{Rate of Convergence}
	Rate of convergence is evaluated using an \emph{error function} $e: \R^n \to \R$ satisfying $e(x)\geq 0$ for all $x\in \R^n$ and $e(x^*)=0$. Typical choices are $e(x)=\norm{x-x^*}$ and $e(x)=\abs{f(x)-f(x^*)}$.

	\begin{definition}[linear convergence]
		We say that $\{e(x^k)\}$ converges \emph{linearly} if there exist $q>0$ and $\beta\in (0,1)$ such that for all $k$, $e(x^k)\leq q \beta^k$.
	\end{definition}
	 It is possible to show that linear convergence is obtained if for some $\beta \in (0,1)$ we have
	 $$
	 \limsup_{k\to\infty}\frac{e(x^{k+1})}{e^{k}}\leq \beta.
	 $$
	 \begin{definition}[superlinear convergence]
	 	If for every $\beta\in (0,1)$, there exists $q$ such that the condition $e(x^k)\leq q\beta^k$ holds for all $k$, we say that $\{e(x^k)\}$ converges \emph{superlinearly}.
	 \end{definition}
	This is true in particular, if 
	$$
	\lim_{k\to\infty}\frac{e(x^{k+1})}{e(x^k)}=0.
	$$
	
	Consider 
	$$f(x)=\frac{1}{2}x^\T Q x,$$ where $Q$ is positive definite and symmetric. Let $m$, $M$ be the smallest and biggest eigenvalue of $Q$, respectively.
	\begin{proposition}
		Suppose we use the method of the steepest descent 
		$$
		x^{k+1}=x^k -\alpha^k \grad f(x^k),
		$$
		where the stepsize $\alpha^k$ is chosen to be constant, i.e. $\alpha^k \triangleq \alpha$. Then
		$$
		\frac{\norm{x^{k+1}}}{\norm{x^k}}\leq \frac{M-m}{M+m}.
		$$
		
		If the stepsize $\alpha^k$ is chosen according to the minimization rule
		$$
		\alpha^k=\arg\min_{\alpha\geq 0}f(x^k-\alpha \grad f(x^k)).
		$$
		Then, for all $k$,
		$$
		f(x^{k+1})\leq \left(\frac{M-m}{M+m}\right)^2 f(x^k).
		$$
	\end{proposition}
	\appendix
	\bibliographystyle{alpha}
	\bibliography{opt_bib.bib} 
\end{document}
